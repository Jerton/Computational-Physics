\documentclass[UTF8]{ctexart}

\usepackage{amsmath}
\usepackage{cases}
\usepackage{cite}
\usepackage{graphicx}
\usepackage{float} 
\usepackage{subfigure}
\usepackage{pifont}
\usepackage{multirow}
\usepackage{diagbox}
\usepackage[margin=1in]{geometry}
\geometry{a4paper}
\usepackage{fancyhdr}
\usepackage{indentfirst}
\usepackage{booktabs}
\usepackage{url}
\pagestyle{fancy}
\usepackage{subfig}
\usepackage{tabularray}
\usepackage{lmodern}
\fancyhf{}


\title{计算物理作业1}
\author{季宇桐 2100017820}
\date{\today}

\pagenumbering{arabic}

\setlength{\parindent}{2em} %2em代表首行缩进两个字符

\begin{document}

\maketitle

\section{数值误差的避免}
为便于观察变化趋势,本大题的呈现结果为一表格,代表了$x$从$0$到$100$以步长为$10$增加时,从$0$到$n$求和的结果,每一行代表某一求和上限下不同x值的结果,可以用如下示意表表示,其中result$=\sum_{n=0}^{N}f_n(x)$,生成结果应在表格内容中(具体数据运行程序)。
\begin{table}[hp]
    \centering
    \begin{tblr}{
        vline{-} = {1}{},
        vline{1-2} = {2-8}{},
        hline{1-2} = {-}{},
        hline{3-9} = {1}{},
    }
        \diagbox{$N$}{result}{$x$}   & 0 & 10 & 20 & 30 & … & 90 & 100 \\
        0   & 1 & 1  & 1  & 1  & … & 1  & 1   \\
        1   & 1 &    &    &    &   &    &     \\
        2   & 1 &    &    &    &   &    &     \\
        3   & 1 &    &    & …  & … &    &     \\
        4   & 1 &    &    &    &   &    &     \\
        …   & … &    &    &    &   &    &     \\
        100 & 1 &    &    &    &   &    &     
    \end{tblr}
    \caption{示意表}
\end{table}

\subsection{直接展开法}
本题采用直接展开,分别计算每一项并相加。从程序运行结果可以看到,此方法得到的结果不收敛,并且摆动越来越大,在$n=34$附近就溢出了,用此方法计算不可行,在计算其中的大项时会导致溢出。

\subsection{递归法}
本题编写递归函数并调用。从运行结果来看,由于使用递归规避了大数的阶乘和幂,递归法可以收敛到稳定的值,但是由于存在大数相消效应,导致计算得到的结果误差很大,如对$e^{-10}$的计算收敛到了$-6.25618\text{e}-05$,明显偏离实际值$4.54\times 10^{-5}$。

\subsection{取倒数法}
由于计算$e^x$时全为正数,结果是稳定的,所以可以稳定获得正确的值,取倒数后获得稳定收敛的$e^{-x}$,经验证,所得结果与真实值相符,这是有效可行的算法。

\section{矩阵的模与条件数}
\subsection*{(a)}
依题意,\begin{equation*}
    A=\begin{pmatrix}
        1 & -1 & \cdots& \cdots & -1\\
          &  1 & -1 & \cdots &-1\\
          &    &  \ddots & \vdots &\vdots\\
          &    &         & 1 & -1\\
          & & & & 1\\
    \end{pmatrix}
\end{equation*}

由于这是个上三角矩阵,因此在计算矩阵的行列式时每一列只有对角元有贡献,即\begin{equation*}
    \det(A) = 1^n = 1
\end{equation*}

因此该矩阵是非奇异的。

\subsection*{(b)}
使用初等行变换法求逆矩阵,$(A,I)\to (I,A^{-1})$.将$A$变为单位矩阵时可以使用如下的步骤:先将第$n$行加到第$n-1$行,再把变换后的第$n-1$和第$n$行加到第$n-2$行,以此类推。将相同的初等行变换作用到单位矩阵上,于是得到\begin{equation*}
    A^{-1}=\begin{pmatrix}
        1&1&2&\cdots &2^{n-3} & 2^{n-2}\\
         &1&1&2&\cdots &2^{n-3}\\
         & &\ddots&\ddots& \vdots &\vdots\\
         & &  &1 &1 & 2\\
         & & & & 1&1\\
         & &  & & &1\\
    \end{pmatrix}
\end{equation*}

\subsection*{(c)}
由定义

$||\vec{x}||_\infty=\max_{1\leq i\leq n}|x_i|$,
\begin{equation*}
    \begin{aligned}
        ||A||_\infty&=\sup_{\vec{x}\ne0}\frac{||A\vec{x}||_\infty}{||\vec{x}||_\infty}\\
        &=\sup_{\vec{x}\ne0}\frac{\max_{1\leq i\leq n}\{\sum_{j=1}^{m}|a_{ij}x_j|\}}{\max_{1\leq i\leq n}\{|x_i|\}}\\
        &=\sup_{\vec{x}\ne0}\max_{1\leq i\leq n}\{\sum_{j=1}^{m}|a_{ij}|\frac{|x_j|}{|x|_{\max}}\}
    \end{aligned}
\end{equation*}

要求上确界,在$\vec{x}$的每一个分量都是${|x|_{\max}}$时可以取到,即
\begin{equation*}
    ||A||_\infty=\max_{1\leq i\leq n}\{\sum_{j=1}^{m}|a_{ij}|\}
\end{equation*}

\subsection*{(d)}
对幺正矩阵有$U^\dagger U=1$,因此\begin{equation*}
    \frac{||U\vec{x}||_2}{||\vec{x}||_2}=\frac{(\vec{x}^\dagger U^\dagger U \vec{x})^\frac{1}{2}}{(\vec{x}^\dagger\vec{x})^\frac{1}{2}}=\frac{(\vec{x}^\dagger\vec{x})^\frac{1}{2}}{(\vec{x}^\dagger\vec{x})^\frac{1}{2}}=1
\end{equation*}

同理
\begin{equation*}
    \frac{||U^\dagger\vec{x}||_2}{||\vec{x}||_2}=\frac{(\vec{x}^\dagger U U^\dagger \vec{x})^\frac{1}{2}}{(\vec{x}^\dagger\vec{x})^\frac{1}{2}}=\frac{(\vec{x}^\dagger\vec{x})^\frac{1}{2}}{(\vec{x}^\dagger\vec{x})^\frac{1}{2}}=1
\end{equation*}

对于$||UA||_2$,有
\begin{equation*}
    ||UA||_2=\frac{(\vec{x}^\dagger  A^\dagger U^\dagger U A \vec{x})^\frac{1}{2}}{(\vec{x}^\dagger\vec{x})^\frac{1}{2}}=\frac{(\vec{x}^\dagger  A^\dagger A \vec{x})^\frac{1}{2}}{(\vec{x}^\dagger\vec{x})^\frac{1}{2}}=||A||_2
\end{equation*}

因此,\begin{equation*}
    K_2(UA)=||UA||_2=||A||_2=K_2(A)
\end{equation*}

\subsection*{(e)}
根据(c),最大显然在第一行取到,即
\begin{equation*}
    \begin{aligned}
        &||A||_\infty=\max_{1\leq i\leq n}\sum_{j=1}^{m}|a_{ij}|=n\\
        &||A^{-1}||_\infty=\max_{1\leq i\leq n}\sum_{j=1}^{m}|a^\prime_{ij}|=2^{n-1} \\
        &K_\infty(A)=n2^{n-1}
    \end{aligned}
\end{equation*}

\section{Hilbert矩阵}
\subsection*{(a)}
取极小值要求$\frac{\partial D}{\partial c_k}=0$,即
\begin{equation*}
    \begin{aligned}
        \frac{\partial D}{\partial c_k}&=0\\
        \int_{0}^{1} 2x^{k-1}\left(\sum_{i=1}^{n}c_i x^{i-1}-f(x)\right)\,dx &=0\\
        \int_{0}^{1} \left(\sum_{i=1}^{n}c_i x^{k+i-2}-f(x)x^{i-1}\right)\,dx &=0\\
        \sum_{i=1}^{n}c_i\frac{1}{k+i-1}&=\int_{0}^{1} f(x)x^{k-1} \,dx \\
        \text{左边是哑指标,}j\to i,i\to k\text{可以写作,}\\
        \sum_{j=1}^{n}c_j\frac{1}{i+j-1}&=\int_{0}^{1} f(x)x^{i-1} \,dx
    \end{aligned}
\end{equation*}

对照\begin{equation*}
    \sum_{j=1}^{n}(H_n)_{ij}c_{j}=b_i 
\end{equation*}

得,
\begin{equation*}
    \begin{aligned}
        (H_n)_{ij}&=\frac{1}{i+j-1}\\
        b_i&=\int_{0}^{1} f(x)x^{i-1} \,dx
    \end{aligned}
\end{equation*}

\subsection*{(b)}
由$(H_n)_{ij}=\frac{1}{i+j-1}=\frac{1}{j+i-1}=(H_n)_{ji}$,故$H$是对称的。

\begin{equation*}
    \begin{aligned}
        &\sum_{i=1}^{n}\sum_{j=1}^{n}c_i(H_n)_{ij} c_j=\sum_{i=1}^{n}\sum_{j=1}^{n}\frac{c_i c_j}{i+j-1}\\
        =&\sum_{i=1}^{n}\sum_{j=1}^{n}c_i c_j\int_{0}^{1} x^{i+j-2} \,dx \\
        =&\int_{0}^{1}\left(\sum_{i=1}^{n} c_i x^{i-1}\right)^2 \,dx \geq0
    \end{aligned}
\end{equation*}

当且仅当所有的$c_i=0$时取等,因此该矩阵是正定的。由线性代数知识,正定矩阵合同于单位矩阵,即存在可逆矩阵$C$,使得$H=C^\mathbf{T}IC$,则$\det(H)=\det(C^\mathbf{T}C)=\det(C^\mathbf{T})\det(C)=\det(C)^2$,因此该矩阵是非奇异的。

\subsection*{(c)}
取对数后$\ln \det(H_n)=4\ln c_n -\ln c_{2n}$,由Stirling公式,$\ln (n!)\approx n\ln n -n$,得$\ln \det(H_n)=4\sum_{i=1}^{n-1}(i \ln i -i)-\sum_{i=1}^{2n-1}(i \ln i -i)$,下面给出$n\leq10$的计算结果
\begin{table}[hp]
    \centering
    \begin{tabular}{|l|l|l|l|l|l|l|l|l|l|} 
    \hline
    $n$ & 2     & 3     & 4      & 5      & 6      & 7      & 8      & 9       & 10       \\ 
    \hline
    $\ln \det(H_n)$  & -2.68 & -9.73 & -19.92 & -33.15 & -49.36 & -68.52 & -90.60 & -115.59 & -143.46  \\
    \hline
    \end{tabular}
    \end{table}

    由此可见以非常快的速度指数减小。
\subsection*{(d)}
具体的计算结果见程序运行,两个结果在$n$增大时出现明显差异,我认为GEM方法更准确,因为Hilbert矩阵极度病态,在$n$较大时Cholesky分解中的开平方等运算的误差会被放大,而GEM不涉及开方,相对准确一些。尽管在n较小时Cholesky算法的结果更接近真实值,但是随着$n$增大,Cholesky算法很快失效,这也可以从后几项Cholesky算法结果无法计算看出来。

\section{矩阵与二次型}
\subsection*{(a)}
依题意\begin{equation*}
    \begin{aligned}
        &B-\lambda I=\begin{pmatrix}
            1-\lambda&-1&0\\
            -1&2-\lambda&-1\\
            0&-1&1-\lambda
        \end{pmatrix}  
        &\det(B-\lambda I)=\lambda(\lambda-1)(3-\lambda)=0
    \end{aligned}
\end{equation*}

得到$\lambda_1=1,\lambda_2=3,\lambda_3=0$,对应特征向量计算如下:
\begin{equation*}
    (B-\lambda_1 I)\vec{x}_1=\begin{pmatrix}
        0&-1&0\\
        -1&1&-1\\
        0&-1&0
    \end{pmatrix}
    \begin{pmatrix}
        x_{11}\\x_{12}\\x_{13}
    \end{pmatrix}=0  
\end{equation*}
令$|x_{1}|=1$,得特征值$\lambda=1$的特征向量是$\vec{x}_1=(\frac{1}{\sqrt{2}},0,\frac{-1}{\sqrt{2}})^\mathrm{T}$.同理,\begin{equation*}
    (B-\lambda_2 I)\vec{x}_2=\begin{pmatrix}
        -2&-1&0\\
        -1&-1&-1\\
        0&-1&-2
    \end{pmatrix}
    \begin{pmatrix}
        x_{21}\\x_{22}\\x_{23}
    \end{pmatrix}=0  
\end{equation*}
令$|x_{2}|=1$,得特征值$\lambda=3$的特征向量是$\vec{x}_2=(\frac{1}{\sqrt{6}},\frac{-2}{\sqrt{6}},\frac{1}{\sqrt{6}})^\mathrm{T}$.同理,\begin{equation*}
    (B-\lambda_2 I)\vec{x}_2=\begin{pmatrix}
        1&-1&0\\
        -1&2&-1\\
        0&-1&1
    \end{pmatrix}
    \begin{pmatrix}
        x_{31}\\x_{32}\\x_{33}
    \end{pmatrix}=0  
\end{equation*}
令$|x_{3}|=1$,得特征值$\lambda=0$的特征向量是$\vec{x}_2=(\frac{1}{\sqrt{3}},\frac{1}{\sqrt{3}},\frac{1}{\sqrt{3}})^\mathrm{T}$.

\subsection*{(b)}
根据实对称矩阵对角化的知识有,\begin{equation*}
    \begin{aligned}
        Q&=\begin{pmatrix}
            \frac{1}{\sqrt{2}}&\frac{1}{\sqrt{6}}&\frac{1}{\sqrt{3}}\\
            0&-\frac{2}{\sqrt{6}}&\frac{1}{\sqrt{3}}\\
            -\frac{1}{\sqrt{2}}&\frac{1}{\sqrt{6}}&\frac{1}{\sqrt{3}}
        \end{pmatrix}\\
        \Sigma&=\begin{pmatrix}
            1&0&0\\
            0&3&0\\
            0&0&0
        \end{pmatrix}
    \end{aligned}
\end{equation*}

验证了$QQ^{\mathrm{T}}=I$,$B=Q\Sigma Q^\mathbf{T}$.

对二次型$\frac{1}{2}u^\mathbf{T}Bu$做单位正交变换$\tilde{u}=Q^\mathbf{T}u$,则变换后的二次型是$\tilde{x}^2+3\tilde{y}^2-2=0$,变换后的特征向量为\begin{equation*}
    \vec{x_1}=\begin{pmatrix}
        1\\0\\0
    \end{pmatrix}
    \vec{x_2}=\begin{pmatrix}
        0\\1\\0
    \end{pmatrix}
    \vec{x_3}=\begin{pmatrix}
        0\\0\\1
    \end{pmatrix}
\end{equation*}
 绘制图像如下\begin{figure}[hp]
    \centering
    \includegraphics*[scale=0.3]{pic4_2.png}
 \end{figure}

\section{正定矩阵}
\subsection*{(a)}
\begin{equation*}
    A\vec{x}=\begin{pmatrix}
        x_1\\x_2-x_1\\x_3-x_2\\x_4-x_3\\-x_4
    \end{pmatrix}
\end{equation*}

故$u^\mathbf{T}K_4u=(Au)^\mathbf{T}(Au)=x_1^2+(x_2-x_1)^2+(x_3-x_2)^2+(x_4-x_3)^2+(-x_4)^2\geq0$

若上式等于零,则$x_1=x_2-x_1=x_3-x_2=x_4-x_3=x_4=0$,进而$x_1=x_2=x_3=x_4=0$,与$\vec{x}$是非零向量矛盾,故$u^\mathbf{T}K_4u>0$.
\subsection*{(b)}
首先我们将$S$对角化,先计算其特征多项式,为\begin{equation*}
    \det(S-\lambda I)=\lambda^2-6\lambda+8-b^2
\end{equation*}

由求根公式\begin{equation*}
    \lambda=3\pm\sqrt{1+b^2}
\end{equation*}
由于$S$是实对称矩阵,因此对应的特征向量在归一化后可以组成单位正交矩阵$U$使$S$对角化,即$U^\mathbf{T}SU=\text{diag}\{\lambda_1,\lambda_2\}$.对于任意的非零向量$\vec{u}$,总可以找到$\tilde{u}=U^{-1}u$,则$u^\mathbf{T}Su=\tilde{u}^\mathbf{T}U^\mathbf{T}SU\tilde{u}=\lambda_1\tilde{x_1}^2+\lambda_2\tilde{x_2}^2$.

因此,当$\lambda_1>0,\lambda_2>0$时矩阵是正定的,即$-2\sqrt{2}<b<2\sqrt{2}$;当有一个特征值为$0$时$S$是半正定的,此时$b=\pm 2\sqrt{2}$. 

Gauss消元后有\begin{equation*}
    S\to\begin{pmatrix}
        2&b\\
        &4\frac{b^2}{2}
    \end{pmatrix} 
\end{equation*}

因此主元为$2$和$4-\frac{b^2}{2}$,可以看出当主元为正时原矩阵正定,主元$\ge0$时原矩阵半正定.
\nocite{*}
\bibliographystyle{plain}
\bibliography{./template}

\end{document}
